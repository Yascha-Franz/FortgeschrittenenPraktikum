\subsection{Untergrundsubtraktion}
Es wird ein exponentieller Untergrund angenommen.
\begin{equation}
  I_{Untergrund}(T) = ae^{bT}+c
\end{equation}
Mit den Daten aus Tabelle \ref{tab:2grad} ergeben sich für die $2\si{\kelvin\per\minute}$-Heizrate die Parameter
\begin{align}
  a &= (7,88\pm 2,38)\SI{e-2}{\pico\ampere} \nonumber\\
  b &= (1,74\pm 0,09)\SI{e-2}{\per\kelvin} \nonumber\\
  c &= (-3,27\pm 0,42)\si{\pico\ampere}
\end{align}
und aus Tabelle \ref{tab:1,5grad} für die die $1,5\si{\kelvin\per\minute}$-Heizrate die Parameter
\begin{align}
  a &= (1,48\pm 0,78)\si{\pico\ampere} \nonumber\\
  b &= (7,00\pm 1,25)\SI{e-3}{\per\kelvin} \nonumber\\
  c &= (-6,49\pm 1,75)\si{\pico\ampere}
\end{align}
Die Messwerte und die Ausgleichskurven sind in Abbildung \ref{fig:Mit_Untergrund} und die bereinigten Werte in Abbildung \ref{fig:Bereinigt} dargestellt.
\begin{figure}
  \centering
  \begin{subfigure}{0.4\textwidth}
    \centering
    \includegraphics[width=\textwidth]{build/I_mit_untergrund_2.pdf}
    \subcaption{Heizrate von $2\si{\kelvin\per\minute}$}
  \end{subfigure}
  \begin{subfigure}{0.4\textwidth}
    \centering
    \includegraphics[width=\textwidth]{build/I_mit_untergrund_15.pdf}
    \subcaption{Heizrate von $1,5\si{\kelvin\per\minute}$}
  \end{subfigure}
  \caption{Gemessener Strom. In die Ausgleichsrechnung miteinbezogene Werte sind in schwarz.}
  \label{fig:Mit_Untergrund}
\end{figure}

\begin{figure}
  \centering
  \begin{subfigure}{0.4\textwidth}
    \centering
    \includegraphics[width=\textwidth]{build/I_2.pdf}
    \subcaption{Heizrate von $2\si{\kelvin\per\minute}$}
  \end{subfigure}
  \begin{subfigure}{0.4\textwidth}
    \centering
    \includegraphics[width=\textwidth]{build/I_15.pdf}
    \subcaption{Heizrate von $1,5\si{\kelvin\per\minute}$}
  \end{subfigure}
  \caption{Bereinigter Strom}
  \label{fig:Bereinigt}
\end{figure}

\subsection{Aktivierungsarbeit aus Anlaufkurve}
Da der erste Berg eine Exponentialfunktion ist ergibt sich die Beziehung
\begin{equation}
  \text{ln}(I(T)) = \frac{-W}{k_bT} + b
\end{equation}
Es lassen sich aus den bereinigten Daten die Werte
\begin{align}
  W_1 &= (1,260\pm 0,084)\SI{e-19}{\joule}\nonumber\\
  b &= 38,0\pm 2,4
\end{align}
für die $2\si{\kelvin\per\minute}$-Heizrate und
\begin{align}
  W_2 &= (6,87\pm 0,84)\SI{e-20}{\joule}\nonumber\\
  b &= 21,5\pm 2,4
\end{align}
für die $1,5\si{\kelvin\per\minute}$-Heizrate bestimmen.
Die Werte sind in Abbildung \ref{fig:Anlauf} dargestellt.

\begin{figure}
  \centering
  \begin{subfigure}{0.4\textwidth}
    \centering
    \includegraphics[width=\textwidth]{build/Anlauf_2.pdf}
    \subcaption{Heizrate von $2\si{\kelvin\per\minute}$}
  \end{subfigure}
  \begin{subfigure}{0.4\textwidth}
    \centering
    \includegraphics[width=\textwidth]{build/Anlauf_15.pdf}
    \subcaption{Heizrate von $1,5\si{\kelvin\per\minute}$}
  \end{subfigure}
  \caption{Logarithmierter Strom gegenüber inverser Temperatur}
  \label{fig:Anlauf}
\end{figure}

\subsection{Aktivierungsarbeit aus Integration}
Die Funktion \eqref{eq:ln2} ist ebenfalls exponentialverteilt, bzw.
\begin{equation}
  \text{ln}(f(T)) = \frac{-W}{k_bT} + b
\end{equation}
es ergeben sich die Werte
\begin{align}
  W_3 &= (1,721\pm 0,028)\SI{e-19}{\joule}\nonumber\\
  b &= -46,7\pm 0,8
\end{align}
für die $2\si{\kelvin\per\minute}$-Heizrate und
\begin{align}
  W_4 &= (1,872\pm 0,105)\SI{e-19}{\joule}\nonumber\\
  b &= -51,4\pm 3,0
\end{align}
für die $1,5\si{\kelvin\per\minute}$-Heizrate bestimmen.
Die Werte sind in Abbildung \ref{fig:Integriert} dargestellt.

\begin{figure}
  \centering
  \begin{subfigure}{0.4\textwidth}
    \centering
    \includegraphics[width=\textwidth]{build/Integriert_2.pdf}
    \subcaption{Heizrate von $2\si{\kelvin\per\minute}$}
  \end{subfigure}
  \begin{subfigure}{0.4\textwidth}
    \centering
    \includegraphics[width=\textwidth]{build/Integriert_15.pdf}
    \subcaption{Heizrate von $1,5\si{\kelvin\per\minute}$}
  \end{subfigure}
  \caption{$\text{ln}(f(T))$ gegenüber inverser Temperatur}
  \label{fig:Integriert}
\end{figure}

\subsection{Relaxationszeit}
Mit den Werten für die Aktivierungsarbeit ergeben sich nach der Formel \eqref{eq:tau_0} die Werte
\begin{align}
  \tau_{0,1} &= (2,04\pm 4,89)\SI{e-15}{\second}\nonumber\\
  \tau_{0,2} &= (3,74\pm 9,24)\SI{e-8}{\second}\nonumber\\
  \tau_{0,3} &= (3,83\pm 3,05)\SI{e-21}{\second}\nonumber\\
  \tau_{0,4} &= (0,49\pm 1,48)\SI{e-22}{\second}\label{eq:relax}
\end{align}