Für etwa 900 Sekunden wird an den Plattenkodensator ein E-Feld mit einer Spannung von 950 V angelegt. Dabei sollte
der Ionenkristall eine Temperatur von T = 320 K (bei KBr) haben.
Ist der Plattenkondensator aufgeladen, wird die Probe auf 210 K abgekühlt. Anschließend wird das E-Feld abgestellt und
der Kondensator wird für 5 Minuten kurzgeschlossen.
Dann wird mit dem Picoamperemeter der Strom beobachtet. Ab einem konstanten Strom von etwa 1-2 pA wird die Probe wieder
gleichmäßig auf 330 K erwärmt. Dabei werden Temperatur und Depolarisationsstrom in Abhängigkeit von der Zeit gemessen.

Der Kristall soll für zwei verschiedene Heizraten $b$ untersucht werden.
