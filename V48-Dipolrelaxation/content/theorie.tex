Werden zweiwertige Kationen (hier Sr$^{2+}$) in ein Gitter, das ansonsten aus einwertigen Ionen besteht gebracht, so entstehen permanente
elektrische Dipole. Dies liegt daran, dass nach dem Einsetzen eines Sr$^{2+}$-Ions eine Leerstelle an einem der Gitterplätze
erzeugt wird, um die Ladungsneutralität zu bewahren. Das Ion und die Leerstelle bilden dann einen Dipol.

Die Richtung dieses Dipols hängt von der Verbindungsachse zwischen den beiden Störstellen ab. Dabei sind nur diskrete
Ausrichtungen möglich.
Will der Dipol seine Richtung ändern, so geschieht dies nur über sogenannte Leerstellendiffusion (bei niedrigeren Temperaturen $T$ unter
500°C können sich nur Leerstellen hin und her bewegen). Dazu muss jedoch eine bestimmte Aktivierungsenergie $W$ vorhanden sein,
um eine bestimmte, durch das Gitter festgelegte, Potentialschwelle überwinden zu können.

Aufgrund von thermischen Bewegungen ist ein Teil der Dipolgesamtheit dazu in der Lage. Dieser beträgt nach Boltzmann

\begin{equation}
	\mathrm{e}^{-\frac{W}{k_B T}} .
\end{equation}

Die Relaxationszeit ergibt sich proportional dazu aus

\begin{equation}
	\tau = \tau_0 \mathrm{e}^{\frac{W}{k_B T}} .
\end{equation}

Dies ist die mittlere Zeit zwischen zwei Umorientierungen eines Dipols. Dabei ist $\tau_0 = \tau(\infty)$ die charakteristische Relaxationszeit.

Die Richtungen der Dipole sind ohne ein angelegtes Feld statistisch verteilt. Befindet sich die Probe aber in einem
elektrischen Feld, so drehen sich die Dipole alle in Richtung des Feldes. Die thermische Bewegung der Gitterbausteine
stört diesen Prozess jedoch.

Der Teil der Dipolgesamtheit, der letztendlich in Feldrichtung zeigt, kann in diesem Versuch näherungsweise durch

\begin{equation}
	y(T) = \frac{p E}{3 k_B T}
\end{equation}

beschrieben werden. Dabei ist $E$ die Feldstärke des elektrischen Feldes und $p$ der Betrag des Dipolmomentes.

Der Teil $y(T)$ stell sich jedoch nur ein, wenn das elektrische Feld lange genug ($t \gg \tau$) eingeschaltet ist.
Um einen solchen Polarisationszustand beizubehalten, sollte die Temperatur der Probe möglichst schnell auf eine Temperatur $T_0$
herabgesenkt werden. Dann kann das E-Feld ausgeschaltet werden, ohne dass der Zustand y(T) großartig verändert wird.
$T_p$ ist dann die Polarisationstemperatur.

Wird der Kondensator anschließend kurzgeschlossen und die Probe mit möglichst konstanter Heizrate $b$ erhitzt, so nehmen
die Dipole wieder nach und nach eine statistische Verteilung an. Dies ist die sogenannte Dipolrelaxation, die einen
Depolarisationsstrom $i(T)$ hervorruft.

\begin{equation}
	i(T) = F y(T_p) p \frac{\mathrm{d} N}{\mathrm{d} t} \label{eq:i1}
\end{equation}

Dabei ist $F$ der Probenquerschnitt und $\frac{\mathrm{d} N}{\mathrm{d} t}$ die Rate der pro Volumen- und Zeiteinheit
relaxierenden Dipole.

Die Änderung der noch ausgerichteten Dipole mit der Zeit kann durch die DGL

\begin{equation}
	\frac{\mathrm{d} N}{\mathrm{d} t} = -\frac{N}{\tau(T)}
\end{equation}

beschrieben werden. Die Zahl der ausgerichteten Dipole hängt also von der Relaxationszeit bzw. von der
Temperatur ab.

Durch das Lösen und Einsetzen der DGL in \eqref{eq:i1} ergibt sich der Zusammenhang zwischen dem Strom und der Temperatur.

%\begin{equation}
%	i(T) = F \frac{p^2 E}{3 k_B T_p} \frac{N_0}{\tau_0} \mathrm{e}^{-\frac{1}{b \tau_0} \int_{T_0}^T \mathrm{e}^{-\frac{W}{k_B T'}} \mathrm{d} T'} \cdot \mathrm{e}^{-\frac{W}{k_B T}}
%\end{equation}

%Für niedrige Temperaturen kann die Näherung

%\begin{equation}
%	i_(T) \approx F \frac{p^2 E}{3 k_B T_p} \frac{N_p}{\tau_0} \mathrm{e}^{-\frac{W}{k_B T}}
%\end{equation}

%benutzt werden.
Unter Verwendung des Logarithmus ergibt sich mit $i_0 = 1$ pA

\begin{equation}
	\ln\left(\frac{i}{i_0}\right) = -\frac{W}{k_B} \cdot \frac{1}{T} + const . \label{eq:ln1}
\end{equation}


Die Polarisation $P$ im Kristall ändert sich ebenfalls. Diese Änderung kann durch

\begin{equation}
  \frac{\mathrm{d} P}{\mathrm{d} t} = - \frac{P}{\tau(T)}
\end{equation}

beschreiben werden.
Dadurch ergibt sich ein Strom

\begin{equation}
  i(t) = F \frac{\mathrm{d} P}{\mathrm{d} t} .
\end{equation}

Daraus kann wieder eine Beziehung zwischen dem Strom und der Temperatur hergeleitet werden.

\begin{equation}
	\ln\left(\frac{I(T)}{i(T) \cdot const}\right) = \frac{W}{k_B} \cdot \frac{1}{T} \label{eq:ln2} .
\end{equation}

Dabei ist $I(T)$ ein Integral

\begin{equation}
	\int_{T}^{T^*} i(T') \mathrm{d}T'
\end{equation}

wobei $T^*$ eine hinreichend große Temperatur ist, bei der der Strom auf nahezu Null abgefallen ist.
Die Aktivierungsenergie $W$ kann durch das Auftragen der logarithmischen Terme in \eqref{eq:ln1} oder \eqref{eq:ln2} gegen
1/T erhalten werden. Die Steigung der erhaltenen Geraden entspricht dann nämlich $W/k_B$.

Aus der Position des Maximums $T_{max}$ der Kurve des Polarisationsstroms kann anschließend die Relaxationszeit $\tau_0$
bestimmt werden.

\begin{equation}
	\tau_0 = \frac{k_B T^2_{max}}{W b} \mathrm{e}^{-\frac{W}{k_B T_{max}}} .
\end{equation}
