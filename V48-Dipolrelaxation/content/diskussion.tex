Der Literaturwert\cite{literaturwert} für die Aktivierungsenergie in Strontium dotierten Kalium-Bromid Kristallen liegt bei
\begin{equation}
  W_{lit} = \SI{0.66}{\electronvolt}.
\end{equation}
Bei Betrachtung der bestimmten Aktivierungsenergien (Tabelle \ref{tab:alen})
fällt auf, dass beide Methoden auf ein Ergebnis ähnlicher Größenordnung führen.
Allerdings fällt auf, dass die durch Ausgleichsrechnung des Anstiegs des Relaxationsstromes bestimmten Aktivierungsenergien unter denen durch Integration bestimmten liegen.
Beide Methoden liegen insgesamt um den Literaturwert verteilt, auch wenn keiner der Werte mit seinen Unsicherheitsintervallen auf dem Literaturwert liegt.
\begin{table}[H]
  \centering
  \caption{Aktivierungsenergien aus beiden Messmethoden.}
  \label{tab:alen}
  \begin{tabular}{c|c|c|c}
    Anlaufkurve&&Integration&\\
    \hline
    $W_1$ & $W_2$ & $W_3$ & $W_4$\\
    \hline
    $(0,786\pm 0,052)\si{\electronvolt}$ & $(0,429\pm 0,052)\si{\electronvolt}$ & $(1,074\pm 0,017)\si{\electronvolt}$ & $(1,169\pm 0,065)\si{\electronvolt}$ \\
  \end{tabular}
\end{table}

In der Literatur \cite{literaturwert} ist die Relaxationszeit eines Strontium Dipols durch
\begin{equation}
  \tau_{lit} = 4\cdot10^{-14}\si{\second}
\end{equation}
gegeben.
Die berechneten charakteristischen Relaxationszeiten \ref{eq:relax} weisen sehr starke Abweichungen vom Literaturwert auf.
Dies ist vor allem darauf zurückzuführen,
dass sich auch schon sehr kleine Fehler in den Aktivierungsenergien stark auf die errechnete Relaxationszeit fortpflanzen,
da diese Fehler in die Formel \eqref{eq:tau_0} exponentiell eingehen.

Verfahrensunabhängige Fehlerquellen sind hierbei die Schwankung der Heizraten $H$,
welche unter anderem dadurch bedingt sind,
dass zu Beginn des Heizvorganges das Temperaturgefälle zwischen Probe und Umgebung sehr stark ist,
weshalb die Temperatur ohne zu heizen bereits sehr rapide steigt,
wohingegen ab ca. $\SI{50}{\celsius}$ die maximal zur Verfügung stehende Heizleistung aufgebracht werden muss,
um die Heizrate nicht zu sehr fallen zu lassen.
Zudem ist das verwandte Picoampermeter erschütterungssensitiv.
