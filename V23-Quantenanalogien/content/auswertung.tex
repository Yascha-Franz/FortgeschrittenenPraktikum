Für die Fehlerrechung wird die empirische Standartabweichung
\begin{equation}
  \sigma = \sqrt{\frac{1}{n-1} \cdot \sum_{i=1}^n(x_i-\overline{x})^2}
  \label{eqn:Stdabweichung}
\end{equation}
und die Gaußsche Fehlerfortpflanzung
\begin{equation}
  u_y = \sqrt{\sum_{i=1}^n\left(\frac{\delta y}{\delta x_i}u_x\right)^2}
  \label{eqn:gauß}
\end{equation}
verwendet.

\subsection{Zylinderresonator}
\subsubsection{Schallgeschwindigkeit}
Mittels der Gleichung \eqref{eq:resonanz} lässt sich für die Daten aus Tabelle \ref{tab:zylinder} ein Fit der der Form
\begin{equation}
  \Delta f = \frac{c}{2L}
\end{equation}
herleiten. Dieser wird mit der curve\_fit Funktion von scipy \cite{scipy} gefittet und ergibt die Schallgeschwindigkeit
\begin{equation}
  c_{Messung} =(339,33\pm 0,54)\si{\metre\per\second}.
\end{equation}
und den Graphen in Abbildung \ref{fig:plot_zylinder}.
Dieser Wert liegt nahe an einem Theoriewert für Raumtemperatur ($20°$C) und normalem Atmosphärendruck.
\begin{equation}
  c_{Theorie} =343\si{\metre\per\second}
\end{equation}


\begin{figure}
  \centering
  \includegraphics{build/plot.pdf}
  \caption{Frequenzdifferenz in Abhängigkeit der Länge eines Zylinderresonators}
  \label{fig:plot_zylinder}
\end{figure}