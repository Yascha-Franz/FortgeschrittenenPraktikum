Für die Fehlerrechung wird die empirische Standartabweichung
\begin{equation}
  \sigma = \sqrt{\frac{1}{n-1} \cdot \sum_{i=1}^n(x_i-\overline{x})^2}
  \label{eqn:Stdabweichung}
\end{equation}
und die Gaußsche Fehlerfortpflanzung
\begin{equation}
  u_y = \sqrt{\sum_{i=1}^n\left(\frac{\delta y}{\delta x_i}u_x\right)^2}
  \label{eqn:gauß}
\end{equation}
verwendet.

\subsection{Zylinderresonator}
\subsubsection{Schallgeschwindigkeit}
Mittels der Gleichung \eqref{eq:resonanz} lässt sich für die Daten aus Tabelle \ref{tab:zylinder} ein Fit der der Form
\begin{equation}
  \Delta f = \frac{c}{2L}
\end{equation}
herleiten. Dieser wird mit der curve\_fit Funktion von scipy \cite{scipy} gefittet und ergibt die Schallgeschwindigkeit
\begin{equation}
  c_{Messung} =(339,33\pm 0,54)\si{\metre\per\second}.
\end{equation}
und den Graphen in Abbildung \ref{fig:plot_zylinder}.
Dieser Wert liegt nahe an einem Theoriewert für Raumtemperatur ($20°$C) und normalem Atmosphärendruck (Beides konnte nicht gemessen werden und wird daher angenommen).
\begin{equation}
  c_{Theorie} =343\si{\metre\per\second}
\end{equation}


\begin{figure}
  \centering
  \includegraphics{build/plot.pdf}
  \caption{Frequenzdifferenz in Abhängigkeit der Länge eines Zylinderresonators}
  \label{fig:plot_zylinder}
\end{figure}

\subsubsection{Resonanzsweep in Zylinderketten}
Der durchgeführte Resonanzsweep (1-10 kHz) zeigt ähnliche Ergebnisse beim Oszilloskop und PC, wie in Abbildung \ref{fig:plot_2zylinder_sweep} zu erkennen ist.
Die Bilder des PC's sind aber von deutlich besserer Qualität, weshalb diese bei allen weiteren Messungen zur Auswertung verwendet werden.
Die, in den Abbildungen \ref{fig:plot_2zylinder_sweep} und \ref{fig:plot_zylinder_sweep} dargestellten, Resonanzsweeps bestätigen die Resonanzbedingung \eqref{eq:resonanz} aus der Theorie durch ihre stetig ansteigende Anzahl an Resonanzpeaks.
\begin{figure}
  \centering
  \begin{subfigure}{0.48\textwidth}
    \centering
    \includegraphics[width=0.9\textwidth]{Bilder/Oszillator_Zylindermessung/scope_0.png}
    \caption{Am Oszilloskop}
  \end{subfigure}
  \begin{subfigure}{0.48\textwidth}
    \centering
    \includegraphics[width=0.9\textwidth]{Bilder/PC_Zylindermessung/2_Zylinder.png}
    \caption{Am PC}
  \end{subfigure}
  \caption{Resonanzsweep bei 2 50mm Zylindern am Oszilloskop und am PC}
  \label{fig:plot_2zylinder_sweep}
\end{figure}

\begin{figure}
  \centering
  \begin{subfigure}{0.32\textwidth}
    \centering
    \includegraphics[width=0.9\textwidth]{Bilder/PC_Zylindermessung/4_Zylinder.png}
    \caption{4 Zylinder}
  \end{subfigure}
  \begin{subfigure}{0.32\textwidth}
    \centering
    \includegraphics[width=0.9\textwidth]{Bilder/PC_Zylindermessung/8_Zylinder.png}
    \caption{8 Zylinder}
  \end{subfigure}
  \begin{subfigure}{0.32\textwidth}
    \centering
    \includegraphics[width=0.9\textwidth]{Bilder/PC_Zylindermessung/12_Zylinder.png}
    \caption{12 Zylinder}
  \end{subfigure}
  \caption{Resonanzsweep von mehreren 50mm Zylindern am PC}
  \label{fig:plot_zylinder_sweep}
\end{figure}

\subsection{Kugelresonator}
\subsubsection{Resonanzfrequenzen}
Die mit dem Oszilloskop aufgenommenen Resonanzfrequenzen stehen in Tabelle \ref{tab:kugel_resonanz}.
Der im gleichen Bereich aufgenommene Frequenzsweep mit dem PC aus Abbildung \ref{fig:kugel_resonanz} weist ein paar mehr Peaks auf,
die aber relativ nahe an anderen Peaks liegen und dadurch nicht mit dem Oszilloskop aufgelöst werden konnten.

\begin{figure}
  \centering
  \includegraphics[width=0.8\textwidth]{Bilder/PC_Kugelresonator/180_100-10000Hz.png}
  \caption{Resonanzspektrum des Kugelresonators bei 100-10000Hz}
  \label{fig:kugel_resonanz}
\end{figure}

\subsubsection{Winkelabhängigkeit}
Da unser Messwinkel $\alpha$ nicht senkrecht zur Symmetrieachse liegt, muss dieser umgeformt werden können um ihn in die Kugelflächenfunktionen einzusetzen.
Die Beziehung erfolgt aus trigonometrischen Überlegungen und ergibt sich als 
\begin{align}
  \cos(\theta)&=\frac{1}{2}\left(\cos(\alpha)-1\right)\nonumber\\
  \theta&=\arccos\left(\frac{1}{2} \left(\cos(\alpha)-1\right)\right).\label{eq:alpha_zu_theta}
\end{align}
Die ersten Kugelflächenfunktionen (normiert nach max$(Y)=1$) ergeben sich als
\begin{align}
  Y_{00} &= 1\\
  Y_{10} &= \text{cos}(\theta)\\
  Y_{20} &= \frac{1}{2}(3\text{cos}^2(\theta)-1) \\
  Y_{30} &= \frac{1}{2}(5\text{cos}^3(\theta)-3\text{cos}(\theta))\\
  Y_{40} &= \frac{1}{8}(35\text{cos}^4(\theta)-30\text{cos}^2(\theta)+3)\\
  Y_{50} &= \frac{1}{8}(\text{cos}(\theta)\cdot(63\text{cos}^4(\theta)-70\text{cos}^2(\theta)+15))
\end{align}
Werden nun die Daten aus Tabelle \ref{tab:kugel_resonanz_winkel} nach Gleichung \eqref{eq:alpha_zu_theta} umgeformt, ebenfalls nach max$(A)=1$ normiert
und in einem Polarplot geplottet, ergeben sich die Plots aus Abbildung \ref{fig:kugel_resonanz_winkel}.
Die zugeordeneten Kugelflächenfunktionen sind hierbei per Auge gewählt worden und haben aufgrund der Normierung keine Fit-Parameter.

\begin{figure}
  \centering
  \begin{subfigure}{0.32\textwidth}
    \centering
    \includegraphics[width=\textwidth]{build/Kugel_Winkel_3.pdf}
    \caption{$3,666$kHz}
  \end{subfigure}
  \begin{subfigure}{0.32\textwidth}
    \centering
    \includegraphics[width=\textwidth]{build/Kugel_Winkel_6.pdf}
    \caption{$6,174$kHz}
  \end{subfigure}
  \begin{subfigure}{0.32\textwidth}
    \centering
    \includegraphics[width=\textwidth]{build/Kugel_Winkel_7.pdf}
    \caption{$7,379$kHz}
  \end{subfigure}
  \caption{Normierte Amplituden in Abhängigkeit vom Winkel \theta}
  \label{fig:kugel_resonanz_winkel}
\end{figure}

\subsubsection{Peakaufspaltung}
Durch Einsetzen eines Ring in den Ring Kugelresonator entsteht eine Vorzugsrichtung und die Entartung in m wird aufgehoben.
In Abbildung \ref{fig:kugel_ring} ist dies durch ein Aufspaltung des Peaks bei $2,3$kHz erkennen.
Die Aufspaltung wird breiter je dicker der Ring zwischen den Kugelhälften ist.
\begin{figure}
  \centering
  \begin{subfigure}{0.4\textwidth}
    \centering
    \includegraphics[width=\textwidth]{Bilder/PC_Kugelresonator/180_2000-2500_ohneRing.png}
    \caption{Ohne Ring}
  \end{subfigure}
  \begin{subfigure}{0.4\textwidth}
    \centering
    \includegraphics[width=\textwidth]{Bilder/PC_Kugelresonator/180_2000-2500_3mmRing.png}
    \caption{3mm Ring}
  \end{subfigure}
  \begin{subfigure}{0.4\textwidth}
    \centering
    \includegraphics[width=\textwidth]{Bilder/PC_Kugelresonator/180_2000-2500_9mmRing.png}
    \caption{9mm Ring}
  \end{subfigure}
  \begin{subfigure}{0.4\textwidth}
    \centering
    \includegraphics[width=\textwidth]{Bilder/PC_Kugelresonator/180_2000-2500_12mmRing.png}
    \caption{12mm Ring}
  \end{subfigure}
  \caption{Aufspaltung des Resonanzpeaks bei $2,3$kHz}
  \label{fig:kugel_ring}
\end{figure}

Mit der Messung aus Tabelle \ref{tab:kugel_ring} lassen sich die Abbildungen \ref{fig:kugel_ring} erstellen.
Die Resonanz bei $2,146$kHz lässt sich der Funktion $Y_{00}$ zuordnen und die Resonanz bei $2,257$kHz  der Funktion $Y_{01}$.
\begin{figure}
  \centering
  \begin{subfigure}{0.4\textwidth}
    \centering
    \includegraphics[width=\textwidth]{build/Kugel_Ring_21.pdf}
    \caption{$2,146$kHz}
  \end{subfigure}
  \begin{subfigure}{0.4\textwidth}
    \centering
    \includegraphics[width=\textwidth]{build/Kugel_Ring_22.pdf}
    \caption{$2,257$kHz}
  \end{subfigure}
  \caption{Winkelverteilung bei dem aufgespaltenem Peak um $2,2$kHz}
  \label{fig:kugel_ring}
\end{figure}