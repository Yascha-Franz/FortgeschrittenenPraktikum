Die durch unseren Fit erhaltene Schallgeschwindigkeit liegt nahe an einem Theoriewert für 20°C.
Wir konnten aber leider nicht die Temperatur messen, daher ist 20°C nur ein Schätzwert und kann somit auch nur als grobe Orientierung verwendet werden.

Die aufgenommen Resonanzen des Kugelresonators mit PC und Oszilloskop stimmen gut überein, bis auf ein paar schmale Peaks neben deutlich dominanteren, die nicht mit dem Oszilloskop aufgelöst werden konnten.

Bei der Winkelabhängigen Messung der Druckamplitude stimmen die Messungen bei $3,666$kHz und $7,379$kHz sehr gut mit den dazugehörigen Kugelflächenfunktionen überein.
Lediglich bei sind größere Abweichungen in der Amplitude zu beobachten, aber die grobe Form stimmt dennoch gut überein.

Die Messungen bei der Aufspaltung der Peaks durch einen Zwischenring stimmen die Amplituden ebenfalls sehr stark mit den zugehörigen Kugelflächenfunktionen überein.
Durch die Änderung unserer ausgewerten Achse von $\theta$ zu $\alpha$ bleibt die Quantenzahl $m=0$.
Aus Sicht der $\theta$-Achse wäre dies ein klarer Symmetriebruch und damit $m\neq 0$ gewesen.

Die starke Abschwächung des zweiten Peaks legt nahe, dass dies der Bindende Zustand ist.
Es wird die Bindung immer mehr abgeschächt, durch größeren Abstand zwischen den zwei Resonatoren.
Das Analogon wäre, dass zwei Atome bei größerer Entfernung schwächer gebunden sind.
Es lassen sich jedoch bei beiden Zuständen $m=0$ beobachten, was nahelegt, dass antibindender und bindender Zustand unterschiedliche Überlagerung derselben Zustände sind.

Bei den Zylinderketten mit Blenden entspricht jede Blende einem Atom.
Der Durchmesser der Blenden entspricht hierbei der Ladung der Atome bzw. der Kopplung durch diese Atome.
Eine reine Kette ist somit einem reinen 1d Kristall gleichzusetzen, eine alternierende Blendenkopplung entspricht einer zwei-atomigen Basis.
Die Fehlstellen sind analog zu Fehlstellen in einem Kristall, ebenso wie ihre Auswirkung auf die Bandstrukturen.