Zunächst wird der Elektromagnet geeicht. Dazu wird mit einer Hallsonde die Abhängigkeit des Magnetfeldes von der Stromstärke
vermessen.

Mit einem Objektiv und der Linse $L_1$ wird das Licht der Cd-Lampe scharf auf den ersten Spalt $S_1$ abgebildet. Die Linse $L_2$
wird so eingestellt, dass ein paralleles Lichtbündel auf das Prisma fällt. Der Duchmesser des Lichtbündels sollte
dabei nicht größer als der des Prismas sein, da ansonsten Strahlungsverluste auftreten.

Mit der Linse $L_3$ wird ein scharfes Bild auf den nächsten Spalt $S_2$ abgebildet. Dieser wird dazu benutzt, eine bestimmte
Wellenlänge auszusuchen. Es wird zuerst die rote Wellenlänge ausgesucht und dann der Polarisationsfilter so eingestellt,
dass nur der $\sigma$-Anteil des Lichtes hindurchgeht.

Die Linse $L_4$ wird so eingestellt, dass sie ein scharfes Bild auf die Lummer-Gehrcke-Platte abbildet.

Das entstehende Interferenzmuster wird mit der Kamera aufgenommen. Es wird dabei ein Bild mit und ein Bild ohne angeschaltetes
Magnetfeld gemacht.

Anschließend wird der blaue Anteil des Spektrums untersucht, wozu der Spalt $S_2$ neu eingestellt wird. Der Polarisationsfilter
wird so eingestellt, dass zunächst der $\sigma$- und anschließend der $\pi$-Anteil betrachtet werden kann.
Ansonsten wird genauso wie bei der roten Linie vorgegangen.
