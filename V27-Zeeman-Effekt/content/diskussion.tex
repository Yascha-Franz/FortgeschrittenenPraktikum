\begin{table}
    \centering
    \caption{Vergleich gemessener und berechneter Werte der Lande-Faktoren}
    \label{tab:Endergebnisse}
    \begin{tabular}{ccc}
        \toprule
        Linie & $g_{j,theorie}$ & $g_{j,gemessen}$\\
        \midrule
        rot             & $1$       & $0,970\pm 0,022$\\
        blau($\sigma$)  & $0,5$     & $0,548\pm 0,010$\\
        blau($\pi$)     & $1,75$    & $1,78\pm 0,06$\\
        \bottomrule
    \end{tabular}
\end{table}
Die theoretisch berechneten Lande-Faktoren befinden sich innerhalb oder nahe an den gemessenen Werten mit ihren Fehlerintervallen.
Bedenkt man hierzu noch, dass die Ableseunsicherheiten der $\Delta s$ und $\delta s$ nicht mit in die Rechnung eingingen,
und die Kamera zwischen Bildern sich minimal bewegt, kann die Theorie hier zumindest annähernd bestätigt werden.