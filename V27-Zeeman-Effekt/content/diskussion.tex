\begin{table}
    \centering
    \caption{Vergleich gemessener und berechneter Werte der Lande-Faktoren}
    \label{tab:Endergebnisse}
    \begin{tabular}{ccc}
        \toprule
        Linie & $g_{j,theorie}$ & $g_{j,gemessen}$\\
        \midrule
        rot             & $1$       & $0,97\pm 0,05$\\
        blau($\sigma$)  & $0,5$     & $0,55\pm 0,05$\\
        blau($\pi$)     & $1,75$    & $1,78\pm 0,08$\\
        \bottomrule
    \end{tabular}
\end{table}
Die theoretisch berechneten Lande-Faktoren befinden sich innerhalb der 1-$\sigma$-Intervallen der gemessenen Werten.
Das bedeutet, dass die Theorie mit einer Signifikanz von mindestens $16\%$ bestätigt werden kann.