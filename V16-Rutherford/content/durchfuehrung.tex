Zunächst wird die Streukammer evakuiert und anschließend die Sperrspannung des Detektors auf 12V eingestellt.
Die vorverstärkten Pulse auf dem Oszilloskop werden beobachtet, wobei insbesondere Pulshöhe und Anstiegszeit betrachtet werden.
Danach wird die Pulshöhe der detektierten Pulse in Abhängigkeit vom Druck in der Streukammer gemessen.
Dabei wird die mittlere Pulshöhe ermittelt, wozu das Nachleuchten am Oszilloskop eingeschaltet werden sollte.

Anschließend wird der differentielle Wirkungsquerschnitt für die dünne Goldfolie bestimmt, indem eine Messung der Zählraten
in Abhängigkeit vom Streuwinkel durchgeführt wird.

Danach wird die Abhängigkeit des Streuquerschnitts von der Dicke der Folie gemessen, wozu die
$\SI{2}{\micro\meter}$-Goldfolie gegen eine $\SI{4}{\micro\meter}$-dicke Goldfolie ausgetauscht wird.

Zuletzt wird die Z-Abhängigkeit der Intensität der $\alpha$-Strahlung untersucht. Dafür werden Folien aus einem anderen
Material als Gold (Platin und Bismut) in den Strahlengang gebracht und wieder die Intensitäten in Abhängigkeit des Streuwinkels
gemessen.
