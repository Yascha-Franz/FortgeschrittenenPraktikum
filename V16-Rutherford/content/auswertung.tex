%Für die Fehlerrechung wird die empirische Standartabweichung
%\begin{equation}
%  \sigma = \sqrt{\frac{1}{n-1} \cdot \sum_{i=1}^n(x_i-\overline{x})^2}
%  \label{eqn:Stdabweichung}
%\end{equation}
%und die Gaußsche Fehlerfortpflanzung
%\begin{equation}
%  u_y = \sqrt{\sum_{i=1}^n\left(\frac{\delta y}{\delta x_i}u_x\right)^2}
%  \label{eqn:gauß}
%\end{equation}
%verwendet.
%\begin{figure}
%  \centering
%  \includegraphics{build/plot.pdf}
%  \caption{Plot}
%  \label{fig:plot}
%\end{figure}
%
\subsection{Dichteabhängigkeit}
Aus den abgelesenen Minima und Maxima in den Tabellen \ref{tab:dichteprofil} und \ref{tab:dichteprofil_Au} lassen sich die Mittelwerte bestimmen.
Die Maximal- und Minimalwerte werden im Folgenden als die Fehlerintervallgrenzen gewählt.
Der Energieverlust durch das durchqueren der Folie ist proportional zur Amplitudendifferenz der Spannungspulse bei $p=0$.
Da dieser Wert aufgrund endlicher Pumpleistung nicht erreicht werden kann,
wird ein linearer Fit der Form
\begin{equation}
  A(p) = A_0 + bp
\end{equation}
mit der curve\_Fit-Funktion von scipy \cite{scipy} durch die Messwerte beider Messreihen gelegt.
Dieser ergibt die Parameter
\begin{align}
  A_{0,\text{ohne}} &= (12,1 \pm 1,1)\si{\volt}\\
  b_{\text{ohne}}   &= (-3,40 \pm 0,56)\cdot\SI{e-2}{\volt\per\milli\bar}\\
  A_{0,\text{mit}} &= (8,2 \pm 0,9)\si{\volt}\\
  b_{\text{mit}}   &= (-2,22 \pm 0,56)\cdot\SI{e-2}{\volt\per\milli\bar}
\end{align}
und die Plots in Abbildung \ref{fig:dichteprofil}.
Durch den linearen Zusammenhang zur Energie lässt sich der Energieverlust durch die Folie bestimmen
\begin{align}
  \Delta E &= E_{\alpha}\left(1-\frac{A_{0,\text{mit}}}{A_{0,\text{ohne}}}\right)\\
           &= (1,8 \pm 0,5)\si{\mega\electronvolt}
\end{align}
Damit nun die Bethe-Bloch-Gleichung verwendet werden kann, muss noch die mittlere Geschwindigkeit der $\alpha$-Teilchen ermittelt werden.
Dies wird aus der mittleren Energie und der Energie-Geschwindigkeit-Beziehung hergeleiletet.
\begin{align}
  \overline{E} &= \frac{1}{2} E_{\alpha}\left(1 + \frac{A_{0,\text{mit}}}{A_{0,\text{ohne}}} \right)\\
  \overline{v} &= \sqrt{\frac{2\overline{E}}{m_{\alpha}}}\\
               &= (1,49 \pm 0,04)\cdot\SI{e+7}{\meter\per\second}
\end{align}
Desweiteren werden ein paar materialspezifische Größen benötigt (größtenteils entnommen aus \cite{Gold} oder Allgemeinwissen).
Diese sind die Kernladungszahl von Gold $Z=79$ und des $\alpha$-Teilchens $z=2$, die Dichte von Gold $\rho=19,32\, \si{\gram\per\meter\cubed}$
und die Atommasse von Gold $m_G=196,97\,\text{u}$ und die hierraus folgende Atomdichte
\begin{equation}
  n=\frac{\rho}{m_G}=5,9\cdot\SI{e+28}{\per\meter\cubed}
\end{equation}
, sowie die mittlere Ionisierungsenergie $I=Z\cdot 10\si{\electronvolt}$.
Mit diesen Werten und der Gleichung \eqref{eqn:DeltaX} kann nun die Foliendicke
\begin{equation}
  \Delta x = (3,9 \pm 1,1)\cdot\SI{e-6}{\meter}
\end{equation}

\begin{figure}
  \centering
  \includegraphics{build/dichteprofil.pdf}
  \caption{Durchschnittliche Amplitude der Pulse (Balken gehen von Minimum bis Maximum) gegenüber dem Druck in der Streukammer. Schwarz ist ohne Folie, Blau ist mit Folie}
  \label{fig:dichteprofil}
\end{figure}

\subsection{Streuung}
Für die Streuung ergibt aufgrung des differentiellen Wirkungsquerschnitts aus Gleichung \eqref{eqn:dsdo} die Fit-Funktion
\begin{equation}
  N(\Theta) = \frac{C}{\sin^4\left(\frac{\Theta-\Theta_0}{2}\right)}
\end{equation}
und es ergeben sich mit den Daten aus den Tabellen \ref{tab:streu_gold}, \ref{tab:streu_platin} und \ref{tab:streu_bismut}
die Parameter aus Tabelle \ref{tab:Fitparameter}, sowie die Plots aus den Abbildungen \ref{fig:streu_gold}, \ref{fig:streu_platin} und \ref{fig:streu_bismut}.
Für das Gold wurden hierbei nur die Zählraten unter $4\si{\becquerel}$ in den Fit miteinbezogen.
\begin{table}
  \centering
  \caption{Fitparameter der Zählrate-Winkel-Messung}
  \label{tab:Fitparameter}
  \begin{tabular}{c|ccc}
    \toprule
    \diagdown & Gold & Platin & Bismut\\
    \midrule
    $C$         & (2,14 \pm 0,69)\cdot\SI{e-6}{\becquerel} & (1,9 \pm 1,2)\cdot\SI{e-4}{\becquerel} & (8,89 \pm 2,30)\cdot\SI{e-5}{\becquerel} \\
    $\Theta_0$  & 1,65 \pm 0,11                       & 7,19 \pm 1,23                     & 5,54 \pm 0,39                       \\
    \bottomrule
  \end{tabular}
\end{table}
\begin{figure}
  \centering
  \includegraphics{build/streu_gold.pdf}
  \caption{Zählrate in Abhängigkeit des Winkels bei Streuung an einer $2\si{\micro\meter}$ Goldfolie}
  \label{fig:streu_gold}
\end{figure}

\begin{figure}
  \centering
  \includegraphics{build/streu_platin.pdf}
  \caption{Zählrate in Abhängigkeit des Winkels bei Streuung an einer $2\si{\micro\meter}$ Platinfolie}
  \label{fig:streu_platin}
\end{figure}

\begin{figure}
  \centering
  \includegraphics{build/streu_bismut.pdf}
  \caption{Zählrate in Abhängigkeit des Winkels bei Streuung an einer $2\si{\micro\meter}$ Bismutfolie}
  \label{fig:streu_bismut}
\end{figure}

\subsection{Z-Abhängigkeit}
Wird nun die $Z^2$-Abhängigkeit von Gleichung \eqref{eqn:dsdo} überprüft muss der Fitparameter $C$ aus Tabelle \ref{tab:Fitparameter}
gegen der Ordnungszahl $Z$ aufgetragen und mit einem Fit der Form
\begin{equation}
  C(Z) = aZ^2
\end{equation}
gefittet werden.
Es entsteht der Plot in Abbildung \ref{fig:ordnung} und der Parameter
\begin{equation}
  a = (1,54 \pm 0,78)\cdot\SI{e-8}{\becquerel}.
\end{equation}
\begin{figure}
  \centering
  \includegraphics{build/ordnung.pdf}
  \caption{Fit-Parameter $C$ gegenüber der Ordnungszahl $Z$}
  \label{fig:ordnung}
\end{figure}