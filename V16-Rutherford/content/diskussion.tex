Die angegeben Dicke der Goldfolie ist $2\si{\micro\meter}$.
Dies ist im $2\sigma$-Intervall unserer Messung von $\Delta X = (3,9 \pm 1,1)\si{\micro\meter}$ und kann somit zwar nicht verworfen werden,
aber auch nicht bestätigt werden. Diese Abweichung kann aber auch durch mehrere systematische Probleme aufgetreten sein.
Im Speziellen musste die mittlere Höhe der Spannungspulse über Minimal- und Maximal-Wert abgeschätzt werden und kann somit bei einer
asymmetrischen Verteilung stark von unserer Schätzung abweichen.
Eine nicht-ideale Ausrichtung der Goldfolie kann auch die Weglänge durch das Gold verlängern und somit unsere Messung verfälschen.

Der Verlauf unserer Messraten / des Wirkungsquerschnitts weicht beim Gold nahe $\Theta_0$ stark von der Theorie Rutherfords ab.
Dies kann zum Teil an einer Saturierung unseres Detektors liegen,
ist aber eher dem unphysikalischen Divergieren der Theorie zuzuschreiben.
Bei Bismut und Platin haben wir leider nicht in der Nähe von $\Theta_0$ gemessen,
können aber, genau wie beim Gold, die Übereinstimmung mit der Theorie bei etwas von $\Theta_0$ verschiedenen Winkeln bestätigen.

Die $Z^2$-Abhängigkeit des Wirkungsquerschnitts ließ sich leider nicht bestätigen.
Dies liegt aber vermutlich an der geringen Menge an verschiedenen ausgemessenen Materialien und den großen Unsicherheiten beim Platin.