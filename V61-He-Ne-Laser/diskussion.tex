\subsection{Stabilitätsbedingung}
Es scheint, dass unser relativ instabiler Aufbau (insbesondere die Tische) eine Bestätigung der Stabilitätsbedingungen der Spiegel unmöglich gemacht hat.
Es wurden stattdessen die nicht quantitiv beschriebene Stabilität der Umgebung überlagert mit der Stabilitätsbedingung der Spiegel gemessen.
Dies macht eine quantitive Auswertung und damit eine Bestätigung/Wiederlegung der Theorie bei der Konkav:Konkav Anordnung leider unmöglich.
Bei der Plan:Konkav Anordnung lässt sich zwar ein Fit legen, dieser hat aber sehr große Fluktuationen und kann somit nur bedingt als Beleg der Theorie gewählt werden.

\subsection{Polarisation}
Die Polarisation des erzeugten Laserlichts konnte mit unserer Messung eindeutig als linear polarisiert nachgewiesen werden.
Die Brewster-Fenster funktionieren somit einwandfrei, wie in der Theorie beschrieben.
%Ich weiß nicht was ich sonst dazu schreiben soll

\subsection{TEM-Moden}
Die $TEM_{00}$-Mode konnte einwandfrei gemessen und bestätigt werden.
Die $TEM_{01}$-Mode hingegen musste, wie bereits im vorigen Kapitel beschrieben, korrigiert werden, da weder Zentrierung noch Ausrichtung unserer Messgeräte garantiert werden können.
Ein exakte Beschreibung der dadurch entstehenden Formeln ist zwar möglich, erzeugt aber beim Fitten zu viele Parameter, sodass dieses dann leider felschlägt.
Es wurde sich daher mit der linear abhängigen Korrektur von $I_0$ begnügt, welche hinreichend gute Ergebnisse liefert und somit die Theorie bestätigt.

\subsection{Wellenlänge}
Die theoretisch berechnete Wellenlänge von $\lambda=632\si{\nano\meter} $ liegt im Fehlerintervall unserer Messung und kann somit als zumindest annähernd korrekt betrachtet werden.
