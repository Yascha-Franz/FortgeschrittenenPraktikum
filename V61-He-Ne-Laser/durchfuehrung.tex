\subsection{Justage}

Zuerst wird der Laser wie im vorigen Kapitel beschrieben aufgebaut. Zwei Beugungsblenden werden mit maximalem Abstand
zueinander auf die Schiene gestellt. Der Justierlaser und das Laserrohr sind bereits vorjustiert, sodass
nur noch die beiden Resonatorspiegel justiert werden müssen. Dazu müssen die Schrauben an den Spiegeln so eingestellt werden,
dass der Laserstrahl genau mittig auf das Fadenkreuz der Blenden trifft.

Sind alle Komponenten eingestellt, so wird der Justierlaser ausgeschaltet. Als nächstes wird der Strom eingeschaltet und das
Gasgemisch somit zur Entladung gebracht. Wenn alles richtig justiert ist, setzt anschließend die Lasertätigkeit ein. Andernfalls
muss die Justage von Neuem begonnen werden.

\subsection{Stabilitätsbedingung}

Zur Überprüfung der Stabilität eines Lasers (einmal konkav/konkav und einmal plan/konkav) wird dieser auf maximale Intensität
eingestellt. Dies kann mit der Photodiode überprüft werden. Der Abstand der Resonatorspiegel wird von 60 cm an schrittweise um 2 cm
(bzw. 1 cm beim plan/konkaven Resonator) erhöht. Dabei müssen die Spiegel bei jedem neu eingestellten Abstand neu justiert
werden, sodass die Laserleistung bei jeder Einstellung möglichst groß wird.
Die Wertepaare (Länge des Resonators und Intensität des Strahls) werden notiert.

\subsection{TEM-Moden}

Die Länge des Resonators wird wieder auf etwa 60 cm zurückgestellt.
Zur Betrachtung der TEM00- und der TEM01-Mode wird der Draht in den Strahl gebracht und der Strahl mit einer Streulinse vergrößert. Auf einem
Schirm können die unterschiedlichen Moden betrachtet werden. Anschließend wird die Intensität der Moden in Abhängigkeit
von der Position der Photodiode gemessen.

\subsection{Polarisation}

Anstelle des Drahtes und der Streulinse wird ein Polarisationsfilter in den Strahl gebracht. Anschließend wird wieder die Intensität
mit Hilfe der Photodiode gemessen - dieses Mal in Abhängigkeit von der Polarisationsrichtung des Filters. Dabei wird der
Filter schrittweise um 15° gedreht und die jeweilige Intensität notiert.

\subsection{Wellenlänge}

Der Polarisationsfilter wird gegen ein Gitter ausgetauscht und es werden auf einem Schirm die Abstände der Hauptmaxima gemessen.
Außerdem werden der Abstand zwischen Gitter und Photodiode und die Gitterkonstante notiert.
