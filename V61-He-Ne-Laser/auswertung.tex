Anmerkung: Wenn von Intensität gesprochen und Strom gemessen wird, sollte im Hinterkopf behalten werden, dass der gemessene Strom linear zur einfallenden Intensität ist.

Für die Fehlerrechung wird die empirische Standartabweichung
\begin{equation}
  \sigma = \sqrt{\frac{1}{n-1} \cdot \sum_{i=1}^n(x_i-\overline{x})^2}
  \label{eqn:Stdabweichung}
\end{equation}
und die Gaußsche Fehlerfortpflanzung
\begin{equation}
  u_y = \sqrt{\sum_{i=1}^n\left(\frac{\delta y}{\delta x_i}u_x\right)^2}
  \label{eqn:gauß}
\end{equation}
verwendet.

\subsection{Überprüfung der Stabilitätsbedingung}
Mittels der Formel \eqref{eqn:Stabilität} ergeben sich für unsere Spiegel Stabilitätsbedingungen, wie in Abbildung \ref{fig:Stabilität_Theorie} dargestellt. 
Die mit diesen Spiegeln verbundene erreichbare Intensität sollte einen ähnlichen Verlauf besitzen.
Wie aber an den Abbildungen \ref{fig:Stabilität_kon_kon} und \ref{fig:Stabilität_kon_flach} zu erkennen, ist dies nicht gelungen.
Es ist jedoch ersichtlich, dass der flache Spiegel deutlich schlechtere Resultate liefert, als die konakaven Spiegel.

\begin{figure}
  \centering
  \includegraphics{build/Justage_Theorie.pdf}
  \caption{Skizze des Versuchsaufbaus}
  \label{fig:Stabilität_Theorie}
\end{figure}


\begin{figure}
  \centering
  \includegraphics{build/Justage_kon_kon.pdf}
  \caption{Skizze des Versuchsaufbaus}
  \label{fig:Stabilität_kon_kon}
\end{figure}

\begin{figure}
  \centering
  \includegraphics{build/Justage_kon_flach.pdf}
  \caption{Skizze des Versuchsaufbaus}
  \label{fig:Stabilität_kon_flach}
\end{figure}

\subsection{Bestimmung der Polarisation}
\subsection{Polarisation}

Bei der Messung der Intensität des Laserstrahls in Abhängigkeit von der Polarisation des Filters ergaben sich die
Messwerte in Tabelle \ref{tab:pol_data}.
Nach Gleichung \ref{eq:pol} wird mit Scipy \cite{scipy} ein nicht-linearer Fit der Form $I = I_0$cos$(\phi+\phi_0)^2$ durchgeführt.
Es ergeben sich Fit-Parameter
\begin{align}
  \phi_0&=(-0,412\pm 0,016)\si{\radian}\nonumber\\
  I_0&=(173,5\pm 3,3)\si{\micro\ampere}\label{eqn:Parameter_Polarisation}
\end{align}
und der Plot aus Abbildung \ref{fig:Polarisation}.
\begin{figure}
  \centering
  \includegraphics{build/Polarisation.pdf}
  \caption{Skizze des Versuchsaufbaus}
  \label{fig:Polarisation}
\end{figure}

%Polarisation: \Theta_0 I_0
%[-0.41170658 173.52359794]
%[0.01643449 3.29294166]

\subsection{Beobachtung der TEM-Moden}
Bei vermessen der $TEM_{00}$-Mode ergeben sich die Messwerte aus Tabelle \ref{tab:mode00_data}. 
Es wird aufgrund der Gleichung \eqref{eq:mode00_fct} ein Fit der Form
\begin{equation}
  I=I_0^2e^{-2\frac{(x-x_0)^2}{\omega^2}}
\end{equation}
gewählt. Wird dieser nun mithilfe der curve\_fit Funktion von scipy \cite{scipy} gefittet und geplottet ergeben sich die Fit-Parameter
\begin{align}
  I_0&=(1,589\pm0,015)\si{\micro\ampere}\nonumber\\
  x_0&=(-0,973\pm0,074)\si{\milli\meter}\nonumber\\
  \omega&=(12,75\pm0,18)\si{milli\meter}
\end{align}
und der Plot aus Abbildung \ref{fig:Mode_00}.
\begin{figure}
  \centering
  \includegraphics{build/Moden_00.pdf}
  \caption{Skizze des Versuchsaufbaus}
  \label{fig:Mode_00}
\end{figure}
Aufgrund der Gleichung \eqref{eq:} wird für die $TEM_{01}$-Mode der Fit 
\begin{equation}
  I=(I_0-a*x)(x-x_0)^2e^{-2\frac{(x-x_0)^2}{\omega^2}}
\end{equation}
gewählt. Der Parameter $a$ ist hierbei die Korrektur darauf, dass wir nicht garantieren können perfekt die $x$-Achse getroffen zu haben.
Es wird also in Erster Näherung eine lineare Abhängigkeit angenommen.
Gefittet mit unseren Messwerten aus Tabelle \ref{tab:mode01_data} und der curve\_fit Funktion von scipy \cite{scipy} ergeben sich die Fit-Parameter 
\begin{align}
  I_0&=(3,01\pm0,18)\si{\micro\ampere}\nonumber\\
  x_0&=(0,35\pm0,23)\si{\milli\meter}\nonumber\\
  \omega&=(13,47\pm0,29)\si{\milli\meter}\nonumber\\
  a&=(0,103\pm0,011)\si{\micro\ampere}
\end{align}
und der Plot aus Abbildung \ref{fig:Mode_01}.

\begin{figure}
  \centering
  \includegraphics{build/Moden_01.pdf}
  \caption{Skizze des Versuchsaufbaus}
  \label{fig:Mode_01}
\end{figure}


\subsection{Bestimmung der Wellenlänge}
Die Wellenlänge des Lasers wird mit Gleichung \ref{eq:wellen} berechnet. Der Wert wird auf der Grundlage vom Abstand $x$ des
nullten zum ersten Hauptmaximum bestimmt, da nicht mehr Hauptmaxima zu erkennen waren.
Dabei wurde $x = 4.6$ cm gemessen.
Die Gitterkonstante ist $g = 0.01$ mm und der Abstand vom Schirm zum Gitter $d = 70$ cm.

Für die Wellenlänge des Lasers ergibt sich somit $\lambda = 656$ nm.
\begin{equation}
  \sigma = \sqrt{\frac{1}{n-1} \cdot \sum_{i=1}^n(x_i-\overline{x})^2}
  \label{eqn:Stdabweichung}
\end{equation}