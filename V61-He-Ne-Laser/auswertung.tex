\subsection{Polarisation}

Bei der Messung der Intensität des Laserstrahls in Abhängigkeit von der Polarisation des Filters ergaben sich die
Messwerte in Tabelle \ref{tab:pol_data}.
Nach Gleichung \ref{eq:pol} wird mit Scipy \cite{scipy} ein nicht-linearer Fit der Form $I/I_0 = a$cos$(bx+c)^2$ durchgeführt.

\begin{table}
\centering
\caption{Intensität des Laserstrahls in Abhängigkeit von der eingestellten Polarisation.}
\label{tab:pol_data}
\begin{tabular}{c c c}
\toprule
$\phi/°$ & $\phi/rad$ & $I/\mu A$ \\
\midrule
0 & 0 & 141.00 \\
15 & 0.262 & 169.00 \\
30 & 0.524 & 164.00 \\
45 & 0.785 & 143.00 \\
60 & 1.047 & 103.00 \\
75 & 1.309 & 66.00 \\
90 & 1.571 & 25.00 \\
105 & 1.833 & 4.10 \\
120 & 2.094 & 1.45 \\
135 & 2.356 & 15.90 \\
150 & 2.617 & 51.00 \\
165 & 2.878 & 95.00 \\
180 & 3.142 & 150.00 \\
195 & 3.403 & 203.00 \\
210 & 3.665 & 186.00 \\
225 & 3.927 & 151.00 \\
240 & 4.189 & 99.00 \\
255 & 4.451 & 69.00 \\
270 & 4.712 & 23.10 \\
285 & 4.974 & 3.25 \\
300 & 5.236 & 1.68 \\
315 & 5.498 & 18.20 \\
330 & 5.760 & 54.00 \\
345 & 6.021 & 102.00 \\
\bottomrule
\end{tabular}
\end{table}


\subsection{Bestimmung der Wellenlänge}

Die Wellenlänge des Lasers wird mit Gleichung \ref{eq:wellen} berechnet. Der Wert wird auf der Grundlage vom Abstand $x$ des
nullten zum ersten Hauptmaximum bestimmt, da nicht mehr Hauptmaxima zu erkennen waren.
Dabei wurde $x = 4.6$ cm gemessen.
Die Gitterkonstante ist $g = 0.01$ mm und der Abstand vom Schirm zum Gitter $d = 70$ cm.

Für die Wellenlänge des Lasers ergibt sich somit $\lambda = 656$ nm.

Für die Fehlerrechung wird die empirische Standartabweichung
\begin{equation}
  \sigma = \sqrt{\frac{1}{n-1} \cdot \sum_{i=1}^n(x_i-\overline{x})^2}
  \label{eqn:Stdabweichung}
\end{equation}
und die Gaußsche Fehlerfortpflanzung
\begin{equation}
  u_y = \sqrt{\sum_{i=1}^n\left(\frac{\delta y}{\delta x_i}u_x\right)^2}
  \label{eqn:gauß}
\end{equation}
verwendet.
